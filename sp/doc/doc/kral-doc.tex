\documentclass[12pt, a4paper]{article}

\usepackage[czech]{babel}
\usepackage{lmodern}
\usepackage[utf8]{inputenc}
\usepackage[T1]{fontenc}
\usepackage{graphicx}
\usepackage{amsmath}
\usepackage[hidelinks,unicode]{hyperref}
\usepackage{float}
\usepackage{listings}
\usepackage{tikz}
\usepackage{xcolor}
\usepackage[final]{pdfpages}
\usepackage{tabularx}

\definecolor{mauve}{rgb}{0.58,0,0.82}
\usetikzlibrary{shapes,positioning,matrix,arrows}

\newcommand{\img}[1]{(viz obr. \ref{#1})}

\definecolor{pblue}{rgb}{0.13,0.13,1}
\definecolor{pgreen}{rgb}{0,0.5,0}
\definecolor{pred}{rgb}{0.9,0,0}
\definecolor{pgrey}{rgb}{0.46,0.45,0.48}

\lstset{frame=tb,
  language=C,
  aboveskip=3mm,
  belowskip=3mm,
  showstringspaces=false,
  columns=flexible,
  basicstyle={\small\ttfamily},
  numbers=none,
  numberstyle=\tiny\color{gray},
  keywordstyle=\color{blue},
  commentstyle=\color{dkgreen},
  stringstyle=\color{mauve},
  breaklines=true,
  breakatwhitespace=true,
  tabsize=3
}

\lstset{language=Java,
  showspaces=false,
  showtabs=false,
  breaklines=true,
  showstringspaces=false,
  breakatwhitespace=true,
  commentstyle=\color{pgreen},
  keywordstyle=\color{pblue},
  stringstyle=\color{pred},
  basicstyle=\ttfamily,
  moredelim=[il][\textcolor{pgrey}]{$$},
  moredelim=[is][\textcolor{pgrey}]{\%\%}{\%\%}
}

\let\oldsection\section
\renewcommand\section{\clearpage\oldsection}

\begin{document}
	% this has to be placed here, after document has been created
	% \counterwithout{lstlisting}{chapter}
	\renewcommand{\lstlistingname}{Ukázka kódu}
	\renewcommand{\lstlistlistingname}{Seznam ukázek kódu}
    \begin{titlepage}

       \centering

       \vspace*{\baselineskip}

       \begin{figure}[H]
          \centering
          \includegraphics[width=7cm]{img/fav-logo.jpg}
       \end{figure}

       \vspace*{1\baselineskip}
       {\sc Semestrální práce z předmětu KIV/UPS}
       \vspace*{1\baselineskip}

       \vspace{0.75\baselineskip}

       {\LARGE\sc Tahová multiplayerová hra na způsob Kris Kros\\}

       \vspace{4\baselineskip}
       
		\vspace{0.5\baselineskip}

       
       {\sc\Large Stanislav Král \\}

       \vspace{0.5\baselineskip}

       {A17B0260P}

       \vfill

       {\sc Západočeská univerzita v Plzni\\
       Fakulta aplikovaných věd}


    \end{titlepage}


    \tableofcontents
    \pagebreak


    
    \section{Úvod}
    \section{Popis hry Kris Kros}
    \section{Analýza}
   	    \section{Popis protokolu}
	    Protokol je \textbf{textový} a jeho zprávy mají speciální formát.
	    \subsection{Formát zpráv}
		Každá zpráva musí začínat řídícím znakem \texttt{\$}. Po počátečním znaku následuje délka datové části zprávy v bytech. Délka zprávy je přirozené nenulové číslo. Po délce zprávy následuje oddělovací znak \texttt{\#}, za kterým se nachází přirozené číslo reprezentující typ zprávy. Typ zprávy je opět následován oddělovacím znakem \texttt{\#},\ za kterým je umístěné přirozené číslo reprezentující identifikátor zprávy. Za identifikátorem zprávy se nachází poslední oddělovací znak \texttt{\#}, který odděluje identifikátor a datovou část zprávy. Datová část je ve formátu \textbf{JSON} a nesmí obsahovat neescapovaný řídící znak \texttt{\$}. Escapování tohoto řídícího znaku se realizuje tím, že se před znak umístí znak \texttt{\textbackslash}.
		
        \begin{lstlisting}[caption={Zpráva protokolu, která je typu \texttt{7} a je identifikována číslem \texttt{1}. Její datová část je dlouhá 19 bytů.},captionpos=b]
		$19#7#1#{"name" : "standa"}
		\end{lstlisting}
			
        \begin{lstlisting}[caption={Zpráva protokolu, která je typu \texttt{6} a je identifikována číslem \texttt{4}. Její datová část je dlouhá 16 bytů.},captionpos=b]
		$16#6#4#{"ready" : true}
		\end{lstlisting}
	

    \section{Implementace}
	    \subsection{Typy příchozích zpráv na serveru}		    
	    
\begin{center}
		\begin{tabular}{| p{1.2cm} | p{3.4cm} | p{7.830cm} |}
			\hline
			\textbf{Typ zprávy} & \textbf{Název zprávy} & \textbf{Popis zprávy} \\ 
			\hline
			2          &CreateLobby              &Požadavek na vytvoření herní místnosti.\\
			\hline
			3          &GetLobbies              &Požadavek na získání aktuálního seznamu herních místností.\\
			\hline
			4          &JoinLobby              &Požadavek na vstoupení do herní místnosti. \\
			\hline
			5          &LeaveLobby              &Požadavek na opuštění herní místnosti.\\
			\hline
			6          &PlayerReady              &Požadavek na změnu připravenosti hráče.\\
			\hline
			7          &UserAuthentication              &Požadavek na autentizaci uživatele\\
			\hline
			8          &UserLeaving              &Požadavek na odpojení uživatele ze se serveru.\\
			\hline
			9          &StartLobby              &Požadavek na spuštění hry.\\
			\hline
			10          &LetterPlaced              &Požadavek na položení písmenka na herní desku.\\
			\hline
			11          &LetterRemoved              &Požadavek na odebrání písmenka z herní desky.\\
			\hline
			12          &FinishRound              &Požadavek na ukončení kola.\\
			\hline
			13          &ApproveWords              &Požadavek na potvrzení aktuálních slov na desce.\\
			\hline
			14          &DeclineWords             &Požadavek na odmítnutí aktuálních slov na desce.\\
			\hline
			15          &KeepAlive              &Požadavek na ověření funkčnosti spojení mezi klientem a serverem.\\
			\hline
			16          &LeaveGame              &Požadavek na opuštění hry.\\
			\hline
		\end{tabular}
\end{center}  

		\subsubsection{Atributy zpráv}
		Níže jsou popsány významy atributů jednotlivých typů zpráv. Pokud se zde nějaká zpráva nevyskytuje znamená to, že nemá žádný atribut.
		
		
\begin{center}
		\begin{table}[!ht]
		     \caption{Atributy zprávy \texttt{JoinLobby}}
		\begin{tabularx}{\textwidth}{|l|l|X|}
			\hline
			\textbf{Název atributu} & \textbf{Typ atributu} & \textbf{Význam atributu} \\ 
			\hline
			\texttt{lobby\_id}          &\texttt{int}&ID místnosti, do které se chce hráč připojit\\
			\hline
		\end{tabularx}
		\end{table}
\end{center}  		
		
\begin{center}
		\begin{table}[!ht]
		     \caption{Atributy zprávy \texttt{PlayerReadyToggle}}
		\begin{tabularx}{\textwidth}{|l|l|X|}
			\hline
			\textbf{Název atributu} & \textbf{Typ atributu} & \textbf{Význam atributu} \\ 
			\hline
			\texttt{ready}          &\texttt{bool}&Booleanská hodnota, která nabývá hodnoty \texttt{true} pokud je hráč připravený a \texttt{false} pokud není připravený.\\
			\hline
		\end{tabularx}
		\end{table}
\end{center}  
		
\begin{center}
		\begin{table}[!ht]
		     \caption{Atributy zprávy \texttt{UserAuthenticated}}
		\begin{tabularx}{\textwidth}{|l|l|X|}
			\hline
			\textbf{Název atributu} & \textbf{Typ atributu} & \textbf{Význam atributu} \\ 
			\hline
			\texttt{name}          &\texttt{string}&Jméno hráče, který se chce autentizovat.\\
			\hline
			\texttt{reconnecting}          &\texttt{bool}&Booleanská hodnota, která nabývá hodnoty \texttt{true} pokud se hráč snaží o znovupřipojení a \texttt{false} pokud ne.\\
			\hline
		\end{tabularx}
		\end{table}
\end{center}  

		
\begin{center}
		\begin{table}[!ht]
		     \caption{Atributy zprávy \texttt{LetterPlaced}}
		\begin{tabularx}{\textwidth}{|l|l|X|}
			\hline
			\textbf{Název atributu} & \textbf{Typ atributu} & \textbf{Význam atributu} \\ 
			\hline
			\texttt{letter}          &\texttt{struct letter}&Písmeno, které chce hráč položit.\\
			\hline
			\texttt{row}          &\texttt{int}&Číslo řádku, na který má být písmenko položené.\\
			\texttt{column}          &\texttt{int}&Číslo sloupce, na který má být písmenko položené.\\
			\hline
		\end{tabularx}
		\end{table}
\end{center}


		
\begin{center}
		\begin{table}[!ht]
		     \caption{Atributy zprávy \texttt{LetterRemoved}}
		\begin{tabularx}{\textwidth}{|l|l|X|}
			\hline
			\textbf{Název atributu} & \textbf{Typ atributu} & \textbf{Význam atributu} \\ 
			\hline
			\texttt{row}          &\texttt{int}&Číslo řádku, ze kterého má být písmenko odebrané.\\
			\texttt{column}          &\texttt{int}&Číslo sloupce, ze kterého má být písmenko odebrané.\\
			\hline
		\end{tabularx}
		\end{table}
\end{center}  		

   	    \subsection{Typy příchozích zpráv na klientovi}
\begin{center}
		\begin{tabularx}{\textwidth}{|l|l|X|}
			\hline
			\textbf{Typ zprávy} & \textbf{Název zprávy} & \textbf{Popis zprávy} \\ 
			\hline
			101          &GetLobbies              &Zpráva obsahující seznam herních místností na serveru.\\
			\hline
			103          &LobbyUpdated         &Zpráva oznamující změnu herní místnosti, která obsahuje aktuální stav místnosti.\\
			\hline
			105          &LobbyDestroyed&Zpráva oznamující zrušení aktuální herní místnosti.\\
			\hline
			106          &LobbyJoined        &Zpráva oznamující úspěšné připojení do herní místnosti.\\
			\hline
			107          &UserAuthenticated            &Zpráva oznamující výsledek autentizace uživatele.\\
			\hline
			108          &LobbyStarted       &Zpráva oznamující spuštění herní místnosti.\\
			\hline
			109          &GameStarted              &Zpráva oznamující spuštění hry.\\
			\hline
			111          &TilesUpdated              &Zpráva obsahující informace o změněných políčkách herní desky a bodový stav hráče, který je momentálně na tahu.\\
			\hline
			112          &RoundFinished              &Zpráva oznamující ukončení aktuálního kola.\\
			\hline
			113          &PlayerAcceptedRound             &Zpráva oznamující to, že některý ze hráčů schválil slovíčka na herní desce.\\
			\hline
			114          &NewRound           &Zpráva oznamující počátek nového kola.\\
			\hline
			115          &YourNewRound&Zpráva oznamující, že začalo nové kolo a uživatel je nyní na tahu\\
			\hline
			116          &PlayerDeclined           &Zpráva oznamující to, že některý ze hráčů zamítl slovíčka na herní desce\\
			\hline
			117          &GameEnded&Zpráva oznamující konec hry, která obsahuje výslednou tabulku bodů jednotlivých hráčů.\\
			\hline
		\end{tabularx}
\end{center}  

\pagebreak

\begin{center}
		\begin{tabularx}{\textwidth}{|l|l|X|}
			\hline
			118          &AcceptResultedInNewRound&Zpráva oznamující to, že schválení slovíček uživatelem vyústilo v nové kolo.\\
			\hline
			119          &PlayerConnectionChanged  &Zpráva oznamující změnu připojení některého z hráčů.\\
			\hline
			120          &GameStateRegeneration              &Zpráva obsahující aktuální stav hry po znovupřipojení k serveru.\\
			\hline
			121          &KeepAlive  &Zpráva potvrzující spojení mezi klientem a serverem.\\
			\hline
			122          &UserStateRegeneration&Zpráva oznamující aktuální stav uživatele po znovupřipojení k serveru.\\
			\hline
			123          &FinishResultedInNextRound&Zpráva oznamující, že potvrzení kola uživatele vyústilo v nové kolo.\\
			\hline
		\end{tabularx}
\end{center}  
Všechny tyto zprávy se vztahují ke kontextu uživatele, kterému přišly.

	    \subsection{Popis serveru}
		    \subsubsection{Struktura modulů}

	    \subsection{Popis klienta}
   		    \subsubsection{Struktura modulů}
    
    \section{Závěr}
    


%obrazek
%\begin{figure}[!ht]
%\centering
%{\includegraphics[width=12cm]{img/poly-example.jpeg}}
%\caption{Zjednodušené UML aplikace (pouze balíčky)}
%\label{fig:photo}
%\end{figure}

	
	

\end{document}    